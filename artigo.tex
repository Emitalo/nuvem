\documentclass[12pt]{article}

\usepackage{sbc-template}

\usepackage{graphicx,url}

\usepackage[brazil]{babel}   
%\usepackage[utf8]{inputenc}
\usepackage[latin1]{inputenc}  

     
\sloppy

\title{Qualidade de Servi�o em servi�os de nuvem: uma an�lise do Openstack}

\author{Emilie Trindade de Morais\inst{1}, Italo Paiva Batista\inst{1}}


\address{Faculdade do Gama, Universidade de Bras�lia
  (UnB)\\
  Gama -- DF -- Brasil
}

\begin{document} 

\maketitle

\begin{abstract}
  This meta-paper describes the style to be used in articles and short papers
  for SBC conferences. For papers in English, you should add just an abstract
  while for the papers in Portuguese, we also ask for an abstract in
  Portuguese (``resumo''). In both cases, abstracts should not have more than
  10 lines and must be in the first page of the paper.
\end{abstract}
     
\begin{resumo} 
  Este meta-artigo descreve o estilo a ser usado na confec��o de artigos e
  resumos de artigos para publica��o nos anais das confer�ncias organizadas
  pela SBC. � solicitada a escrita de resumo e abstract apenas para os artigos
  escritos em portugu�s. Artigos em ingl�s dever�o apresentar apenas abstract.
  Nos dois casos, o autor deve tomar cuidado para que o resumo (e o abstract)
  n�o ultrapassem 10 linhas cada, sendo que ambos devem estar na primeira
  p�gina do artigo.
\end{resumo}


\section{Introdu��o}

\section{Servi�os de nuvem}

\section{Qualidade de Servi�o em nuvem}

\section{Openstack}
\section{Materiais e m�todos}
\section{Desenvolvimento do trabalho}
% \subsection{Subsections}

% \begin{figure}[ht]
% \centering
% \includegraphics[width=.5\textwidth]{fig1.jpg}
% \caption{A typical figure}
% \label{fig:exampleFig1}
% \end{figure}

% \begin{table}[ht]
% \centering
% \caption{Variables to be considered on the evaluation of interaction
%   techniques}
% \label{tab:exTable1}
% \includegraphics[width=.7\textwidth]{table.jpg}
% \end{table}


\bibliographystyle{sbc}
\bibliography{sbc-template}

\end{document}
